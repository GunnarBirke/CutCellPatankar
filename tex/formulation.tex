\documentclass[]{article}

\usepackage{amsmath}
\usepackage{stmaryrd}
\usepackage{tikz}

%opening
\title{}
\author{}

\begin{document}

\maketitle

\section{Current formulation}

So far we have considered the linear transport equation
\begin{align*}
u_t + a u_x & = 0 \quad \forall (t, x) \in (0, T) \times (a, b) \\
u(0, x)  & = u_0(x) \quad \forall x \in (a, b)\\
u(t, a) & = u(t, b) \quad \forall t \in (0, T)
\end{align*}
with $a > 0$, that means we have a wave moving to the right.

Our spatial domain discretizations starts off with $N+1$ equidistant grid points $(x_{i - \frac{1}{2}})_{(1 \le i \le N + 1)}$ on the domain $(a, b)$ forming a grid with $N$ elements  $E_i = (x_{i - \frac{1}{2}}, x_{i + \frac{1}{2}})$ of size $|E_i| = h_i = x_{i + \frac{i}{2}} - x_{i - \frac{1}{2}}$.

Let for some $1 \le k \le N$ a point $x_{cut} \in E_k$  be given, seperating $E_k$ into two cut cells $E_{cut, 1} = (x_{k - \frac{1}{2}}, x_{cut})$ and $E_{cut, 2} = (x_{cut}, x_{k + \frac{1}{2}})$.  This yields the cut cell grid
\[
\mathcal{M}_h = \{ E_i : i \neq k\} \cup \{ E_{cut, 1}, E_{cut, 2} \}.
\]
We will assume that $E_{cut}^1$ is the small cell with $|E_i| \ll h$, requiring additional handling if a timestep size depending on $h$ is chosen.

We define our discrete function space as
\[
V_h = \{ v_h \in L^2((a, b)) : (v_h)_{|E} \in \mathcal{P}^0(E), \; E \in \mathcal{M}_h \}
\]
that means our discrete functions are piecewise constant on the cut cell grid.

For our semi-discretization in space we define
\begin{align*}
g_0(u_h) & = \frac{1}{4} ( u_1 + 2 u_0 + u_N -f(u_1) + f(u_N)) \\
& \vdots \\
g_{k-1} (u_h) & = \frac{1}{4} ( u_{cut, 1} + 2 u_{k-1} + u_{k-2} -f(u_{cut, 1} ) + f(u_{k-2} )) \\
g_{cut, 1} (u_h) & = \frac{1}{4} ( u_{cut, 2}  + 2 u_{cut, 1}  + u_{k-1} -f(u_{cut, 2}) + f(u_{k-1})) \\
g_{cut, 2}(u_h)  & = \frac{1}{4} ( u_{k+1} + 2 u_{cut, 2}  + u_{cut, 1}  -f(u_{k+1}) + f(u_{cut, 1})) \\
g_{k+1}(u_h)  & = \frac{1}{4} ( u_{k+2} + 2 u_{k+1} + u_{cut, 2} -f(u_{k+2}) + f(u_{cut, 2})) \\
& \vdots\\
g_N(u_h)  & = \frac{1}{4} ( u_0 + 2 u_N + u_{N-1} -f(u_0) + f(u_{N-1}))
\end{align*}

Let $\nu_i = \frac{2 \Delta t}{|E_i|}$ and $\beta_i = \frac{|E_i|}{h}$. The fully discrete scheme is given by

\begin{align*}
	u_0^{n+1}& = \frac{(1 - \beta_0 \nu_0) u_0^n + \nu_0 g_0(u_h^n)}{(1 - \beta_0 \nu_0) + \nu_0}\\
	& \vdots \\
	u_{k-1}^{n+1} & = \frac{(1 - \beta_{k-1} \nu_{k-1}) u_{k-1}^n + \nu_{k-1} g_{k-1}(u_h^n)}{(1 - \beta_{k-1} \nu_{k-1}) + \nu_{k-1}}\\
	u_{cut, 1}^{n+1} & = \frac{(1 - \beta_{cut, 1} \nu_{cut, 1}) u_{cut, 1}^n + \nu_{cut, 1} g_{cut, 1}(u_h^n)}{(1 - \beta_{cut, 1} \nu_{cut, 1}) + \nu_{cut, 1}}\\
	u_{cut, 2}^{n+1}  & = \frac{(1 - \beta_{cut, 2} \nu_{cut, 2}) u_{cut, 2}^n + (1 - \beta_{cut, 1}) \nu_{cut, 1} g_{cut, 1} (u_h^n) + \beta_{cut, 1} \nu_{cut, 2} g_{cut, 2}(u_h^n)}{1 - \beta_{cut, 2} \nu_{cut, 2} + (1 - \beta_{cut, 1}) \nu_{cut, 1} + \beta_{cut, 1} \nu_{cut, 2}}\\
	u_{k+1}^{n+1}  & = \frac{(1 - \beta_{k+1} \nu_{k+1}) u_{k+1}^n + \nu_{k+1} g_{k+1}(u_h^n)}{(1 - \beta_{k+1} \nu_{k+1}) + \nu_{k+1}}\\
	& \vdots\\
	u_N^{n+1} & = \frac{(1 - \beta_N \nu_N) u_{n+1}^n + \nu_N g_N(u_h^n)}{(1 - \beta_N \nu_N) + \nu_N}
\end{align*}

\section*{Extend domain of dependence}

The problem that we are currently facing is that we do not properly reconstruct the domain of dependence of elements in the local neighborhood of a small cell. Preferable we would do this on the level of time discretization, similarly to the formulation above, but it's not yet clear how to achieve this.

Extending the time discretization can probably be understood as solving a perturbed ODE anyways (formulate this precisely!), so we might as well work on the level of the space discretization, at least to get some insight. One crucial ingredient for boundedness preservation is the implication
\begin{displaymath}
	u \in G \implies g(u) \in G
\end{displaymath}
or equivalently
\begin{displaymath}
	u _i\in [\alpha, \beta] \forall i \implies g_i(u) \in [\alpha, \beta] \forall i
\end{displaymath}
Here $g_i(u)$ essentially describes the changes of the state $u_i$ due to incoming and outgoing fluxes.

In the formulation above we are essentially considering something like
\begin{displaymath}
(1-\kappa)g_i(u) + \kappa g_{i+1}(u) \in [\alpha, \beta]
\end{displaymath}
for $\kappa \in [0, 1]$, up to Courant numbers. The set relation holds since this is a convex combination. However, this does not seem to use every assumption we have available. In fact, we should have something like
\begin{displaymath}
	g_i(u) + g_{i+1}(u) \in [\alpha, \beta]
\end{displaymath}
due to consistency of the underlying numerical fluxes (check this!). The term $g_i(u) + g_{i+1}(u)$ should desribe  the changes of the combined (discontinuous) state $u_i \oplus u_{i+1}$ in an extended volume. Taking a portion of this, controlled by a parameter $\eta \in [0, 1]$ and applying it via the Patankar scheme to neighboring elements of a small cell should give a reconstruction of the actual domain of dependence.

\section{Note (Meeting 11.10.2024)}
Adapting $g_i(u)$ to track the fluxes and restoring the domain of dependence respectively will result in a scheme euqal (or similar?) to DoD. 
Thereofore, our approach to expand the method to equations with arbirtatry flux directions (for the sake of restoring the domain of dependence) should just be to adjust the appearance of $g_i$'s, like a convex combination, but not adapting the $g_i$'s itself.
\end{document}
